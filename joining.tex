\documentclass[ignorenonframetext,]{beamer}
\setbeamertemplate{caption}[numbered]
\setbeamertemplate{caption label separator}{: }
\setbeamercolor{caption name}{fg=normal text.fg}
\beamertemplatenavigationsymbolsempty
\usepackage{lmodern}
\usepackage{amssymb,amsmath}
\usepackage{ifxetex,ifluatex}
\usepackage{fixltx2e} % provides \textsubscript
\ifnum 0\ifxetex 1\fi\ifluatex 1\fi=0 % if pdftex
  \usepackage[T1]{fontenc}
  \usepackage[utf8]{inputenc}
\else % if luatex or xelatex
  \ifxetex
    \usepackage{mathspec}
  \else
    \usepackage{fontspec}
  \fi
  \defaultfontfeatures{Ligatures=TeX,Scale=MatchLowercase}
\fi
\usetheme[]{metropolis}
% use upquote if available, for straight quotes in verbatim environments
\IfFileExists{upquote.sty}{\usepackage{upquote}}{}
% use microtype if available
\IfFileExists{microtype.sty}{%
\usepackage{microtype}
\UseMicrotypeSet[protrusion]{basicmath} % disable protrusion for tt fonts
}{}
\newif\ifbibliography
\hypersetup{
            pdftitle={Joining dataframes in R using dplyr},
            pdfauthor={Erika Braithwaite, PhD},
            pdfborder={0 0 0},
            breaklinks=true}
\urlstyle{same}  % don't use monospace font for urls
\usepackage{color}
\usepackage{fancyvrb}
\newcommand{\VerbBar}{|}
\newcommand{\VERB}{\Verb[commandchars=\\\{\}]}
\DefineVerbatimEnvironment{Highlighting}{Verbatim}{commandchars=\\\{\}}
% Add ',fontsize=\small' for more characters per line
\usepackage{framed}
\definecolor{shadecolor}{RGB}{248,248,248}
\newenvironment{Shaded}{\begin{snugshade}}{\end{snugshade}}
\newcommand{\KeywordTok}[1]{\textcolor[rgb]{0.13,0.29,0.53}{\textbf{#1}}}
\newcommand{\DataTypeTok}[1]{\textcolor[rgb]{0.13,0.29,0.53}{#1}}
\newcommand{\DecValTok}[1]{\textcolor[rgb]{0.00,0.00,0.81}{#1}}
\newcommand{\BaseNTok}[1]{\textcolor[rgb]{0.00,0.00,0.81}{#1}}
\newcommand{\FloatTok}[1]{\textcolor[rgb]{0.00,0.00,0.81}{#1}}
\newcommand{\ConstantTok}[1]{\textcolor[rgb]{0.00,0.00,0.00}{#1}}
\newcommand{\CharTok}[1]{\textcolor[rgb]{0.31,0.60,0.02}{#1}}
\newcommand{\SpecialCharTok}[1]{\textcolor[rgb]{0.00,0.00,0.00}{#1}}
\newcommand{\StringTok}[1]{\textcolor[rgb]{0.31,0.60,0.02}{#1}}
\newcommand{\VerbatimStringTok}[1]{\textcolor[rgb]{0.31,0.60,0.02}{#1}}
\newcommand{\SpecialStringTok}[1]{\textcolor[rgb]{0.31,0.60,0.02}{#1}}
\newcommand{\ImportTok}[1]{#1}
\newcommand{\CommentTok}[1]{\textcolor[rgb]{0.56,0.35,0.01}{\textit{#1}}}
\newcommand{\DocumentationTok}[1]{\textcolor[rgb]{0.56,0.35,0.01}{\textbf{\textit{#1}}}}
\newcommand{\AnnotationTok}[1]{\textcolor[rgb]{0.56,0.35,0.01}{\textbf{\textit{#1}}}}
\newcommand{\CommentVarTok}[1]{\textcolor[rgb]{0.56,0.35,0.01}{\textbf{\textit{#1}}}}
\newcommand{\OtherTok}[1]{\textcolor[rgb]{0.56,0.35,0.01}{#1}}
\newcommand{\FunctionTok}[1]{\textcolor[rgb]{0.00,0.00,0.00}{#1}}
\newcommand{\VariableTok}[1]{\textcolor[rgb]{0.00,0.00,0.00}{#1}}
\newcommand{\ControlFlowTok}[1]{\textcolor[rgb]{0.13,0.29,0.53}{\textbf{#1}}}
\newcommand{\OperatorTok}[1]{\textcolor[rgb]{0.81,0.36,0.00}{\textbf{#1}}}
\newcommand{\BuiltInTok}[1]{#1}
\newcommand{\ExtensionTok}[1]{#1}
\newcommand{\PreprocessorTok}[1]{\textcolor[rgb]{0.56,0.35,0.01}{\textit{#1}}}
\newcommand{\AttributeTok}[1]{\textcolor[rgb]{0.77,0.63,0.00}{#1}}
\newcommand{\RegionMarkerTok}[1]{#1}
\newcommand{\InformationTok}[1]{\textcolor[rgb]{0.56,0.35,0.01}{\textbf{\textit{#1}}}}
\newcommand{\WarningTok}[1]{\textcolor[rgb]{0.56,0.35,0.01}{\textbf{\textit{#1}}}}
\newcommand{\AlertTok}[1]{\textcolor[rgb]{0.94,0.16,0.16}{#1}}
\newcommand{\ErrorTok}[1]{\textcolor[rgb]{0.64,0.00,0.00}{\textbf{#1}}}
\newcommand{\NormalTok}[1]{#1}

% Prevent slide breaks in the middle of a paragraph:
\widowpenalties 1 10000
\raggedbottom

\AtBeginPart{
  \let\insertpartnumber\relax
  \let\partname\relax
  \frame{\partpage}
}
\AtBeginSection{
  \ifbibliography
  \else
    \let\insertsectionnumber\relax
    \let\sectionname\relax
    \frame{\sectionpage}
  \fi
}
\AtBeginSubsection{
  \let\insertsubsectionnumber\relax
  \let\subsectionname\relax
  \frame{\subsectionpage}
}

\setlength{\parindent}{0pt}
\setlength{\parskip}{6pt plus 2pt minus 1pt}
\setlength{\emergencystretch}{3em}  % prevent overfull lines
\providecommand{\tightlist}{%
  \setlength{\itemsep}{0pt}\setlength{\parskip}{0pt}}
\setcounter{secnumdepth}{0}
\usepackage{tikz}
\usepackage{pgfplots}
\usepackage{graphicx}
\usepackage{bm}
\usepackage{amsmath}
\usepackage{amssymb}
\usepackage{mathtools}
\usetikzlibrary{shadows,shapes,arrows,automata,calc,shapes.geometric,shapes.multipart,positioning,trees}

\title{Joining dataframes in R using dplyr}
\subtitle{RLadies Montreal}
\author{Erika Braithwaite, PhD}
\date{March 15, 2018}

\begin{document}
\frame{\titlepage}

\begin{frame}[fragile]{What is joining?}

We often run into scenarios where we need to join two dataframes
together. Let's say we had some students who were given an IQ test at a
career fair. Some of the students showed up at on both days, but not
all. They were given unique alphanumeric identifiers.

Set up

\begin{Shaded}
\begin{Highlighting}[]
\CommentTok{#install.packages('pacman')}
\NormalTok{pacman}\OperatorTok{::}\KeywordTok{p_load}\NormalTok{(knitr, kableExtra, formattable, data.table, dplyr,  rmarkdown)}
\end{Highlighting}
\end{Shaded}

Make some data

\begin{Shaded}
\begin{Highlighting}[]
\NormalTok{day1 =}\StringTok{  }\KeywordTok{data.table}\NormalTok{(}\DataTypeTok{ID=}\NormalTok{LETTERS[}\DecValTok{1}\OperatorTok{:}\DecValTok{12}\NormalTok{],}
                 \DataTypeTok{IQ=}\KeywordTok{round}\NormalTok{(}\KeywordTok{rnorm}\NormalTok{(}\DecValTok{12}\NormalTok{, }\DecValTok{100}\NormalTok{, }\DecValTok{15}\NormalTok{),}\DecValTok{2}\NormalTok{))}

\NormalTok{day2 =}\StringTok{  }\KeywordTok{data.table}\NormalTok{(}\DataTypeTok{ID=}\NormalTok{LETTERS[}\DecValTok{6}\OperatorTok{:}\DecValTok{17}\NormalTok{],}
                 \DataTypeTok{IQ=}\KeywordTok{round}\NormalTok{(}\KeywordTok{rnorm}\NormalTok{(}\DecValTok{12}\NormalTok{, }\DecValTok{100}\NormalTok{, }\DecValTok{20}\NormalTok{),}\DecValTok{2}\NormalTok{))}
\end{Highlighting}
\end{Shaded}

\end{frame}

\begin{frame}[fragile]{Our data}

\begin{Shaded}
\begin{Highlighting}[]
\KeywordTok{kable}\NormalTok{(}\KeywordTok{list}\NormalTok{(day1, day2), }\StringTok{'html'}\NormalTok{) }\OperatorTok\StringTok{ }
\StringTok{        }\KeywordTok{kable_styling}\NormalTok{(}\DataTypeTok{full_width =}\NormalTok{ F, }\DataTypeTok{font_size =} \DecValTok{14}\NormalTok{)}
\end{Highlighting}
\end{Shaded}

ID

IQ

A

93.01

B

82.54

C

74.53

D

114.50

E

74.33

F

87.11

G

81.42

H

107.71

I

113.88

J

113.76

K

110.13

L

107.80

ID

IQ

F

106.06

G

92.50

H

91.12

I

99.29

J

115.30

K

90.55

L

84.07

M

49.12

N

137.48

O

94.60

P

123.68

Q

100.25

\end{frame}

\begin{frame}[fragile]{Three(ish) types of joins in dplyr}

If we wanted to merge the two dataframes together, we can use dplyr's
joining functions:

\begin{enumerate}
\def\labelenumi{\arabic{enumi}.}
\item
  Mutating join: add a new variables to one table from matching rows in
  another

\begin{verbatim}
left_join
right_join
inner_join
\end{verbatim}
\item
  Filtering join: filter observations from one tables based on whether
  they match

\begin{verbatim}
full_join
semi_join
anti_join
\end{verbatim}
\item
  Set operations:

\begin{verbatim}
intersect
union
setdiff
\end{verbatim}
\end{enumerate}

\end{frame}

\begin{frame}[fragile]{Mutating joins}

\begin{Shaded}
\begin{Highlighting}[]
\NormalTok{left =}\StringTok{ }\KeywordTok{left_join}\NormalTok{(day1, day2, }\DataTypeTok{by =} \StringTok{'ID'}\NormalTok{)}
\NormalTok{right =}\StringTok{ }\KeywordTok{right_join}\NormalTok{(day1, day2, }\DataTypeTok{by =} \StringTok{'ID'}\NormalTok{)}
\NormalTok{inner=}\StringTok{ }\KeywordTok{inner_join}\NormalTok{(day1, day2, }\DataTypeTok{by =} \StringTok{'ID'}\NormalTok{)}
\NormalTok{full =}\StringTok{ }\KeywordTok{full_join}\NormalTok{(day1, day2, }\DataTypeTok{by =} \StringTok{'ID'}\NormalTok{)}
\NormalTok{semi =}\StringTok{ }\KeywordTok{semi_join}\NormalTok{(day1, day2, }\DataTypeTok{by =} \StringTok{'ID'}\NormalTok{)}
\NormalTok{anti =}\StringTok{ }\KeywordTok{anti_join}\NormalTok{(day1, day2, }\DataTypeTok{by =} \StringTok{'ID'}\NormalTok{)}
\end{Highlighting}
\end{Shaded}

ID

IQ.x

IQ.y

F

87.11

106.06

G

81.42

92.50

H

107.71

91.12

I

113.88

99.29

J

113.76

115.30

K

110.13

90.55

L

107.80

84.07

M

NA

49.12

N

NA

137.48

O

NA

94.60

P

NA

123.68

Q

NA

100.25

\end{frame}

\begin{frame}[fragile]{dplyr versus data.table versus base R}

\begin{block}{data.table}

\begin{Shaded}
\begin{Highlighting}[]
\KeywordTok{rbind}\NormalTok{(...,}\DataTypeTok{use.names =}\NormalTok{ T, }\DataTypeTok{fill =}\NormalTok{ F, }\DataTypeTok{idcol =} \OtherTok{NULL}\NormalTok{)}
\end{Highlighting}
\end{Shaded}

\end{block}

\begin{block}{Base R}

\begin{Shaded}
\begin{Highlighting}[]
\KeywordTok{rbind}\NormalTok{(...,}\DataTypeTok{deparse.level =} \DecValTok{1}\NormalTok{, }\DataTypeTok{make.row.names =}\NormalTok{ T)}
\end{Highlighting}
\end{Shaded}

\begin{Shaded}
\begin{Highlighting}[]
\KeywordTok{cbind}\NormalTok{()}
\end{Highlighting}
\end{Shaded}

\begin{verbatim}
## NULL
\end{verbatim}

\end{block}

\end{frame}

\begin{frame}{When joining can get tricky\ldots{}}

\begin{itemize}
\item
  Joining with columns having the same name but different encoding
  (UTF-8 vs.~Latin)
\item
  Joining with columns having different storage types (factors,
  integers, bit64, dates)
\item
\end{itemize}

\end{frame}

\begin{frame}{How to choose?}

Dplyr has come a long way in terms of speed. Recent benchmarking
conducted in Stack Overflow showed that dplyr starts to substantially
lag when there are a large number of groups (\textgreater{}100k).

While this point is contentious, if you've been a long standing
tidyversalist, then you might find data.table's syntax more difficult to
learn.

\end{frame}

\begin{frame}{More!}

There are additional types of joins not covered here: rolling joins,
scaling joins

Rolling joins are used in circumstances where you want to map a
dataframe to another based but you lack a common ID e.g.~merging
Resources

\end{frame}

\begin{frame}{Thank-you}

\small
\textbf{Committee member \& unofficial 3rd supervisor} \newline
Dr.~Gavin Shaddick, University of Bath, Dept Math

\vspace{2mm} \textbf{Supervisors} \newline
Dr.~David Buckeridge, McGill EBOH \newline
Dr.~Sarah Henderson, BCCDC \& UBC SPPH

\vspace{2mm}

Dr.~Finn Lindgren, INLA research \& development team

\vspace{2mm}

British Columbia Centre for Disease Control (BCCDC)

\vspace{2mm}

McGill Surveillance Lab

\vspace{5mm}

\definecolor{tempcolor2}{HTML}{ADCAB4}
\setbeamercolor{tealbox}{bg=tempcolor2!75}

\begin{beamercolorbox}[sep=1mm]{tealbox}
\footnotesize
Morrison KT, Shaddick G, Henderson SB, Buckeridge DL. 2016. `A latent process model for forecasting multiple time series in environmental public health surveillance.' \emph{Statistics in Medicine.} DOI: 10.1002/sim.6904
\end{beamercolorbox}

\end{frame}

\end{document}
